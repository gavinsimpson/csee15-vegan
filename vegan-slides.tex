\documentclass[10pt,ignorenonframetext,compress, aspectratio=169]{beamer}
\usetheme{m}
\usefonttheme[onlymath]{serif}
\setbeamertemplate{caption}[numbered]
\setbeamertemplate{caption label separator}{:}
\setbeamercolor{caption name}{fg=normal text.fg}
\usepackage{amssymb,amsmath}
% mathtools for underbracket
\usepackage{mathtools}
% \usepackage{ifxetex,ifluatex} % m theme needs xelatex
\usepackage{fixltx2e} % provides \textsubscript
% \usepackage{lmodern} kills bullets m theme
% \ifxetex
  \usepackage{fontspec,xltxtra,xunicode}
  \defaultfontfeatures{Mapping=tex-text,Scale=MatchLowercase}
  \newcommand{\euro}{€}
% \else
%   \ifluatex
%     \usepackage{fontspec}
%     \defaultfontfeatures{Mapping=tex-text,Scale=MatchLowercase}
%     \newcommand{\euro}{€}
%   \else
%     \usepackage[T1]{fontenc}
%     \usepackage[utf8]{inputenc}
%     %   \fi
% \fi
% use upquote if available, for straight quotes in verbatim environments
\IfFileExists{upquote.sty}{\usepackage{upquote}}{}
% use microtype if available
\IfFileExists{microtype.sty}{\usepackage{microtype}}{}
\usepackage{color}
\usepackage{fancyvrb}
\newcommand{\VerbBar}{|}
\newcommand{\VERB}{\Verb[commandchars=\\\{\}]}
\DefineVerbatimEnvironment{Highlighting}{Verbatim}{commandchars=\\\{\}}
% Add ',fontsize=\small' for more characters per line
\usepackage{framed}
\definecolor{shadecolor}{RGB}{248,248,248}
\newenvironment{Shaded}{\begin{snugshade}}{\end{snugshade}}
\newcommand{\KeywordTok}[1]{\textcolor[rgb]{0.13,0.29,0.53}{\textbf{{#1}}}}
\newcommand{\DataTypeTok}[1]{\textcolor[rgb]{0.13,0.29,0.53}{{#1}}}
\newcommand{\DecValTok}[1]{\textcolor[rgb]{0.00,0.00,0.81}{{#1}}}
\newcommand{\BaseNTok}[1]{\textcolor[rgb]{0.00,0.00,0.81}{{#1}}}
\newcommand{\FloatTok}[1]{\textcolor[rgb]{0.00,0.00,0.81}{{#1}}}
\newcommand{\CharTok}[1]{\textcolor[rgb]{0.31,0.60,0.02}{{#1}}}
\newcommand{\StringTok}[1]{\textcolor[rgb]{0.31,0.60,0.02}{{#1}}}
\newcommand{\CommentTok}[1]{\textcolor[rgb]{0.56,0.35,0.01}{\textit{{#1}}}}
\newcommand{\OtherTok}[1]{\textcolor[rgb]{0.56,0.35,0.01}{{#1}}}
\newcommand{\AlertTok}[1]{\textcolor[rgb]{0.94,0.16,0.16}{{#1}}}
\newcommand{\FunctionTok}[1]{\textcolor[rgb]{0.00,0.00,0.00}{{#1}}}
\newcommand{\RegionMarkerTok}[1]{{#1}}
\newcommand{\ErrorTok}[1]{\textbf{{#1}}}
\newcommand{\NormalTok}[1]{{#1}}

% Comment these out if you don't want a slide with just the
% part/section/subsection/subsubsection title:
% \AtBeginPart{
%   \let\insertpartnumber\relax
%   \let\partname\relax
%   \frame{\partpage}
% }
% \AtBeginSection{
%   \let\insertsectionnumber\relax
%   \let\sectionname\relax
%   \frame{\sectionpage}
% }
% \AtBeginSubsection{
%   \let\insertsubsectionnumber\relax
%   \let\subsectionname\relax
%   \frame{\subsectionpage}
% }

\setlength{\parindent}{0pt}
\setlength{\parskip}{6pt plus 2pt minus 1pt}
\setlength{\emergencystretch}{3em}  % prevent overfull lines
\setcounter{secnumdepth}{0}

% Textcomp for various common symbols
\usepackage{textcomp}

\usepackage{booktabs}

% Creative Commons Icons
\usepackage[scale=2]{ccicons}

\newenvironment{centrefig}{\begin{figure}\centering}{\end{figure}}
\newcommand{\columnsbegin}{\begin{columns}}
\newcommand{\columnsend}{\end{columns}}
\newcommand{\centreFigBegin}{\begin{columns}\centering}
\newcommand{\centreFigEnd}{\end{figure}}


\title{A Brief Introduction to Vegan}
\author{Gavin L. Simpson}
\date{CSEE 2015 • May 20th 2015}

\begin{document}
\frame{\titlepage}

\section{Basic Ordination}\label{basic-ordination}

\begin{frame}{Unconstrained ordination}

What is \alert{unconstrained}

First we look for major variation, then relate it to environmental
variation

vs.~constrained ordination, where we only want to see what can be
explained by environmental variables of interest

How well do we explain the main patterns in the species data vs how
large are the patterns we can expain with the measured data

\end{frame}

\begin{frame}{Examples of unconstrained ordination}

\begin{itemize}
\itemsep1pt\parskip0pt\parsep0pt
\item
  Principal Components Analysis --- PCA
\item
  Correspondance Analysis --- CA
\item
  Nonmetric Multidimensional Scaling --- NMDS
\end{itemize}

\end{frame}

\begin{frame}[fragile]{Before we get started}

Housekeeping

\begin{Shaded}
\begin{Highlighting}[]
\KeywordTok{setwd}\NormalTok{(}\StringTok{"your/working/dir"}\NormalTok{)}
\end{Highlighting}
\end{Shaded}

\begin{Shaded}
\begin{Highlighting}[]
\KeywordTok{library}\NormalTok{(}\StringTok{"vegan"}\NormalTok{)}
\KeywordTok{data}\NormalTok{(dune)}
\KeywordTok{data}\NormalTok{(dune.env)}
\end{Highlighting}
\end{Shaded}

Data from: Jongman, R.H.G, ter Braak, C.J.F \& van Tongeren, O.F.R.
(1987). Data Analysis in Community and Landscape Ecology. Pudoc,
Wageningen.

\end{frame}

\begin{frame}[fragile]{Before we get started \textbar{} species}

\begin{Shaded}
\begin{Highlighting}[]
\KeywordTok{dim}\NormalTok{(dune)                               }\CommentTok{# number of samples, species}
\end{Highlighting}
\end{Shaded}

\begin{verbatim}
[1] 20 30
\end{verbatim}

\begin{Shaded}
\begin{Highlighting}[]
\KeywordTok{head}\NormalTok{(dune[,}\DecValTok{1}\NormalTok{:}\DecValTok{6}\NormalTok{])}
\end{Highlighting}
\end{Shaded}

\begin{verbatim}
  Achimill Agrostol Airaprae Alopgeni Anthodor Bellpere
1        1        0        0        0        0        0
2        3        0        0        2        0        3
3        0        4        0        7        0        2
4        0        8        0        2        0        2
5        2        0        0        0        4        2
6        2        0        0        0        3        0
\end{verbatim}

\end{frame}

\begin{frame}[fragile]{Before we get started \textbar{} environment}

\scriptsize

\begin{Shaded}
\begin{Highlighting}[]
\KeywordTok{head}\NormalTok{(dune.env, }\DataTypeTok{n=}\DecValTok{3}\NormalTok{)}
\end{Highlighting}
\end{Shaded}

\begin{verbatim}
   A1 Moisture Management      Use Manure
1 2.8        1         SF Haypastu      4
2 3.5        1         BF Haypastu      2
3 4.3        2         SF Haypastu      4
\end{verbatim}

\begin{Shaded}
\begin{Highlighting}[]
\KeywordTok{summary}\NormalTok{(dune.env)}
\end{Highlighting}
\end{Shaded}

\begin{verbatim}
       A1         Moisture Management       Use    Manure
 Min.   : 2.800   1:7      BF:3       Hayfield:7   0:6   
 1st Qu.: 3.500   2:4      HF:5       Haypastu:8   1:3   
 Median : 4.200   4:2      NM:6       Pasture :5   2:4   
 Mean   : 4.850   5:7      SF:6                    3:4   
 3rd Qu.: 5.725                                    4:3   
 Max.   :11.500                                          
\end{verbatim}

\normalsize

\end{frame}

\begin{frame}[fragile]{Basic ordination}

PCA finds linear combinations of the variables that explain the largest
amounts of variance in the data

\scriptsize

\begin{Shaded}
\begin{Highlighting}[]
\NormalTok{(pca <-}\StringTok{ }\KeywordTok{rda}\NormalTok{(dune))}
\end{Highlighting}
\end{Shaded}

\begin{verbatim}
Call: rda(X = dune)

              Inertia Rank
Total           84.12     
Unconstrained   84.12   19
Inertia is variance 

Eigenvalues for unconstrained axes:
   PC1    PC2    PC3    PC4    PC5    PC6    PC7    PC8 
24.795 18.147  7.629  7.153  5.695  4.333  3.199  2.782 
(Showed only 8 of all 19 unconstrained eigenvalues)
\end{verbatim}

\end{frame}

\begin{frame}{Basic ordination}

Vegan has a wrapper function for doing NMDS ordinations using best
practices:

\begin{itemize}
\itemsep1pt\parskip0pt\parsep0pt
\item
  \texttt{metaMDS()}
\end{itemize}

This will do handy things

\begin{itemize}
\itemsep1pt\parskip0pt\parsep0pt
\item
  standardize your data if necessary
\item
  perform rotation to PCs
\item
  scale coordinates in half change units
\end{itemize}

\end{frame}

\begin{frame}[fragile]{Basic ordination and plotting}

\begin{Shaded}
\begin{Highlighting}[]
\NormalTok{dune.bray.ord <-}\StringTok{ }\KeywordTok{metaMDS}\NormalTok{(dune, }\DataTypeTok{distance =} \StringTok{"bray"}\NormalTok{, }\DataTypeTok{k =} \DecValTok{2}\NormalTok{, }\DataTypeTok{trymax =} \DecValTok{50}\NormalTok{)}
\end{Highlighting}
\end{Shaded}

\end{frame}

\begin{frame}[fragile]{Basic ordination and plotting (using all
defaults)}

\begin{Shaded}
\begin{Highlighting}[]
\KeywordTok{plot}\NormalTok{(dune.bray.ord)}
\end{Highlighting}
\end{Shaded}

\begin{center}\includegraphics[width=0.5\linewidth]{vegan-slides_files/figure-beamer/NMDS2-1} \end{center}

\end{frame}

\begin{frame}[fragile]{Basic ordination and plotting (just plots)}

\begin{Shaded}
\begin{Highlighting}[]
\KeywordTok{plot}\NormalTok{(dune.bray.ord, }\DataTypeTok{display =} \StringTok{"sites"}\NormalTok{)}
\end{Highlighting}
\end{Shaded}

\begin{center}\includegraphics[width=0.5\linewidth]{vegan-slides_files/figure-beamer/NMDS3-1} \end{center}

\end{frame}

\begin{frame}[fragile]{Basic ordination and plotting (just species)}

\begin{Shaded}
\begin{Highlighting}[]
\KeywordTok{plot}\NormalTok{(dune.bray.ord, }\DataTypeTok{display =} \StringTok{"species"}\NormalTok{)}
\end{Highlighting}
\end{Shaded}

\begin{center}\includegraphics[width=0.5\linewidth]{vegan-slides_files/figure-beamer/NMDS4-1} \end{center}

\end{frame}

\begin{frame}[fragile]{Site names instead of points}

\begin{Shaded}
\begin{Highlighting}[]
\KeywordTok{plot}\NormalTok{(dune.bray.ord, }\DataTypeTok{display =} \StringTok{"sites"}\NormalTok{, }\DataTypeTok{type =} \StringTok{"text"}\NormalTok{)}
\end{Highlighting}
\end{Shaded}

\begin{center}\includegraphics[width=0.5\linewidth]{vegan-slides_files/figure-beamer/NMDS5-1} \end{center}

\end{frame}

\begin{frame}[fragile]{Site names instead of points}

\scriptsize

\begin{Shaded}
\begin{Highlighting}[]
\KeywordTok{plot}\NormalTok{(dune.bray.ord, }\DataTypeTok{display =} \StringTok{"sites"}\NormalTok{)}
\KeywordTok{set.seed}\NormalTok{(}\DecValTok{314}\NormalTok{) ## make reproducible}
\KeywordTok{ordipointlabel}\NormalTok{(dune.bray.ord, }\DataTypeTok{display =} \StringTok{"sites"}\NormalTok{, }\DataTypeTok{scaling =} \DecValTok{3}\NormalTok{, }\DataTypeTok{add =} \OtherTok{TRUE}\NormalTok{)}
\end{Highlighting}
\end{Shaded}

\begin{center}\includegraphics[width=0.5\linewidth]{vegan-slides_files/figure-beamer/NMDS5-2-1} \end{center}

\normalsize

\end{frame}

\begin{frame}[fragile]{Site names instead of points}

\scriptsize

\begin{Shaded}
\begin{Highlighting}[]
\KeywordTok{plot}\NormalTok{(dune.bray.ord, }\DataTypeTok{display =} \StringTok{"species"}\NormalTok{)}
\KeywordTok{set.seed}\NormalTok{(}\DecValTok{314}\NormalTok{) ## make reproducible}
\KeywordTok{ordipointlabel}\NormalTok{(dune.bray.ord, }\DataTypeTok{display =} \StringTok{"species"}\NormalTok{, }\DataTypeTok{scaling =} \DecValTok{3}\NormalTok{, }\DataTypeTok{add =} \OtherTok{TRUE}\NormalTok{)}
\end{Highlighting}
\end{Shaded}

\begin{center}\includegraphics[width=0.5\linewidth]{vegan-slides_files/figure-beamer/NMDS5-5-1} \end{center}

\normalsize

\end{frame}

\begin{frame}[fragile]{Site names instead of points}

\scriptsize

\begin{Shaded}
\begin{Highlighting}[]
\KeywordTok{plot}\NormalTok{(dune.bray.ord)}
\KeywordTok{set.seed}\NormalTok{(}\DecValTok{314}\NormalTok{) ## make reproducible}
\KeywordTok{ordipointlabel}\NormalTok{(dune.bray.ord, }\DataTypeTok{scaling =} \DecValTok{3}\NormalTok{, }\DataTypeTok{add =} \OtherTok{TRUE}\NormalTok{)}
\end{Highlighting}
\end{Shaded}

\begin{center}\includegraphics[width=0.5\linewidth]{vegan-slides_files/figure-beamer/NMDS5-6-1} \end{center}

\normalsize

\end{frame}

\begin{frame}[fragile]{Larger points}

\begin{Shaded}
\begin{Highlighting}[]
\KeywordTok{plot}\NormalTok{(dune.bray.ord, }\DataTypeTok{display =} \StringTok{"sites"}\NormalTok{, }\DataTypeTok{cex=}\DecValTok{2}\NormalTok{)}
\end{Highlighting}
\end{Shaded}

\begin{center}\includegraphics[width=0.5\linewidth]{vegan-slides_files/figure-beamer/NMDS6-1} \end{center}

\end{frame}

\begin{frame}{Modifying the display of the points with environmental
data}

\begin{itemize}
\itemsep1pt\parskip0pt\parsep0pt
\item
  Color
\item
  Shape
\item
  Size
\end{itemize}

\end{frame}

\begin{frame}[fragile]{Modifying the color of points}

\tiny

\begin{Shaded}
\begin{Highlighting}[]
\NormalTok{colors.vec <-}\StringTok{ }\KeywordTok{c}\NormalTok{(}\StringTok{"red"}\NormalTok{, }\StringTok{"blue"}\NormalTok{, }\StringTok{"orange"}\NormalTok{, }\StringTok{"grey"}\NormalTok{)}
\KeywordTok{plot}\NormalTok{(dune.bray.ord, }\DataTypeTok{display =} \StringTok{"sites"}\NormalTok{, }\DataTypeTok{type =} \StringTok{"n"}\NormalTok{)}
\KeywordTok{points}\NormalTok{(dune.bray.ord, }\DataTypeTok{display =} \StringTok{"sites"}\NormalTok{, }\DataTypeTok{cex=}\DecValTok{2}\NormalTok{, }\DataTypeTok{pch =} \DecValTok{21}\NormalTok{, }
       \DataTypeTok{col =} \NormalTok{colors.vec[dune.env$Management], }
       \DataTypeTok{bg =} \NormalTok{colors.vec[dune.env$Management])}
\KeywordTok{legend}\NormalTok{(}\StringTok{"topright"}\NormalTok{, }\DataTypeTok{legend =} \KeywordTok{levels}\NormalTok{(dune.env$Management), }\DataTypeTok{bty =} \StringTok{"n"}\NormalTok{,}
                      \DataTypeTok{col =} \NormalTok{colors.vec, }\DataTypeTok{pch =} \DecValTok{21}\NormalTok{, }\DataTypeTok{pt.bg =} \NormalTok{colors.vec)}
\end{Highlighting}
\end{Shaded}

\begin{center}\includegraphics[width=0.5\linewidth]{vegan-slides_files/figure-beamer/NMDS7-1} \end{center}

\normalsize

\end{frame}

\begin{frame}[fragile]{Adding other layers}

\scriptsize

\begin{Shaded}
\begin{Highlighting}[]
\KeywordTok{plot}\NormalTok{(dune.bray.ord, }\DataTypeTok{display =} \StringTok{"sites"}\NormalTok{, }\DataTypeTok{cex=}\DecValTok{2}\NormalTok{) }\CommentTok{# just site points}
\end{Highlighting}
\end{Shaded}

\begin{center}\includegraphics[width=0.5\linewidth]{vegan-slides_files/figure-beamer/NMDS11-1} \end{center}

\normalsize

\end{frame}

\begin{frame}[fragile]{Adding other layers}

\scriptsize

\begin{Shaded}
\begin{Highlighting}[]
\KeywordTok{plot}\NormalTok{(dune.bray.ord, }\DataTypeTok{display =} \StringTok{"sites"}\NormalTok{, }\DataTypeTok{cex=}\DecValTok{2}\NormalTok{)}
\KeywordTok{ordihull}\NormalTok{(dune.bray.ord,}\DataTypeTok{groups =} \NormalTok{dune.env$Management, }\DataTypeTok{label =} \OtherTok{TRUE}\NormalTok{) }\CommentTok{# convex hulls}
\end{Highlighting}
\end{Shaded}

\begin{center}\includegraphics[width=0.5\linewidth]{vegan-slides_files/figure-beamer/NMDS12-1} \end{center}

\normalsize

\end{frame}

\begin{frame}[fragile]{Adding other layers}

\scriptsize

\begin{Shaded}
\begin{Highlighting}[]
\KeywordTok{plot}\NormalTok{(dune.bray.ord, }\DataTypeTok{display =} \StringTok{"sites"}\NormalTok{, }\DataTypeTok{cex=}\DecValTok{2}\NormalTok{)}
\KeywordTok{ordihull}\NormalTok{(dune.bray.ord,}\DataTypeTok{groups =} \NormalTok{dune.env$Management, }\DataTypeTok{label =} \OtherTok{TRUE}\NormalTok{, }\DataTypeTok{col =} \StringTok{"blue"}\NormalTok{)}
\KeywordTok{ordispider}\NormalTok{(dune.bray.ord,}\DataTypeTok{groups =} \NormalTok{dune.env$Management, }\DataTypeTok{label =} \OtherTok{TRUE}\NormalTok{)}
\end{Highlighting}
\end{Shaded}

\begin{center}\includegraphics[width=0.5\linewidth]{vegan-slides_files/figure-beamer/NMDS14-1} \end{center}

\normalsize

\end{frame}

\begin{frame}[fragile]{Adding other layers - axes scaling}

\scriptsize

\begin{Shaded}
\begin{Highlighting}[]
\KeywordTok{plot}\NormalTok{(dune.bray.ord, }\DataTypeTok{type =} \StringTok{"n"}\NormalTok{)}
\end{Highlighting}
\end{Shaded}

\begin{center}\includegraphics[width=0.5\linewidth]{vegan-slides_files/figure-beamer/NMDS17-1} \end{center}

\normalsize

\end{frame}

\begin{frame}[fragile]{Adding other layers - axes scaling}

\scriptsize

\begin{Shaded}
\begin{Highlighting}[]
\KeywordTok{plot}\NormalTok{(dune.bray.ord, }\DataTypeTok{type =} \StringTok{"n"}\NormalTok{)}
\KeywordTok{points}\NormalTok{(dune.bray.ord, }\DataTypeTok{display =} \StringTok{"sites"}\NormalTok{, }\DataTypeTok{cex =} \DecValTok{2}\NormalTok{)}
\end{Highlighting}
\end{Shaded}

\begin{center}\includegraphics[width=0.5\linewidth]{vegan-slides_files/figure-beamer/NMDS18-1} \end{center}

\normalsize

\end{frame}

\begin{frame}[fragile]{Adding other layers - axes scaling}

\scriptsize

\begin{Shaded}
\begin{Highlighting}[]
\KeywordTok{plot}\NormalTok{(dune.bray.ord, }\DataTypeTok{display =} \StringTok{"sites"}\NormalTok{, }\DataTypeTok{type =} \StringTok{"n"}\NormalTok{)}
\end{Highlighting}
\end{Shaded}

\begin{center}\includegraphics[width=0.5\linewidth]{vegan-slides_files/figure-beamer/NMDS19-1} \end{center}

\normalsize

\end{frame}

\begin{frame}[fragile]{Adding other layers - axes scaling}

\scriptsize

\begin{Shaded}
\begin{Highlighting}[]
\KeywordTok{plot}\NormalTok{(dune.bray.ord, }\DataTypeTok{display =} \StringTok{"sites"}\NormalTok{, }\DataTypeTok{type =} \StringTok{"n"}\NormalTok{)}
\KeywordTok{points}\NormalTok{(dune.bray.ord, }\DataTypeTok{display =} \StringTok{"sites"}\NormalTok{, }\DataTypeTok{cex =} \DecValTok{2}\NormalTok{)}
\end{Highlighting}
\end{Shaded}

\begin{center}\includegraphics[width=0.5\linewidth]{vegan-slides_files/figure-beamer/NMDS20-1} \end{center}

\normalsize

\end{frame}

\begin{frame}[fragile]{Adding other layers}

\scriptsize

\begin{Shaded}
\begin{Highlighting}[]
\KeywordTok{plot}\NormalTok{(dune.bray.ord, }\DataTypeTok{display =} \StringTok{"sites"}\NormalTok{, }\DataTypeTok{type =} \StringTok{"n"}\NormalTok{)}
\KeywordTok{points}\NormalTok{(dune.bray.ord,}\DataTypeTok{display =} \StringTok{"sites"}\NormalTok{, }\DataTypeTok{cex =} \DecValTok{2}\NormalTok{)}
\KeywordTok{ordispider}\NormalTok{(dune.bray.ord,}\DataTypeTok{groups =} \NormalTok{dune.env$Management, }\DataTypeTok{label =} \OtherTok{TRUE}\NormalTok{)}
\end{Highlighting}
\end{Shaded}

\begin{center}\includegraphics[width=0.5\linewidth]{vegan-slides_files/figure-beamer/NMDS21-1} \end{center}

\normalsize

\end{frame}

\begin{frame}[fragile]{Adding other layers}

\scriptsize

\begin{Shaded}
\begin{Highlighting}[]
\KeywordTok{plot}\NormalTok{(dune.bray.ord, }\DataTypeTok{display =} \StringTok{"sites"}\NormalTok{, }\DataTypeTok{type =} \StringTok{"n"}\NormalTok{)}
\KeywordTok{points}\NormalTok{(dune.bray.ord, }\DataTypeTok{display =} \StringTok{"sites"}\NormalTok{, }\DataTypeTok{cex =} \DecValTok{2}\NormalTok{)}
\KeywordTok{ordiellipse}\NormalTok{(dune.bray.ord,}\DataTypeTok{groups =} \NormalTok{dune.env$Management, }\DataTypeTok{label =} \OtherTok{TRUE}\NormalTok{)}
\end{Highlighting}
\end{Shaded}

\begin{center}\includegraphics[width=0.5\linewidth]{vegan-slides_files/figure-beamer/NMDS22-1} \end{center}

\normalsize

\end{frame}

\begin{frame}[fragile]{Adding other layers}

\scriptsize

\begin{Shaded}
\begin{Highlighting}[]
\KeywordTok{plot}\NormalTok{(dune.bray.ord, }\DataTypeTok{display =} \StringTok{"sites"}\NormalTok{, }\DataTypeTok{type =} \StringTok{"n"}\NormalTok{)}
\KeywordTok{points}\NormalTok{(dune.bray.ord,}\DataTypeTok{display =} \StringTok{"sites"}\NormalTok{, }\DataTypeTok{cex =} \DecValTok{2}\NormalTok{)}
\KeywordTok{ordisurf}\NormalTok{(dune.bray.ord,dune.env$A1, }\DataTypeTok{add =} \OtherTok{TRUE}\NormalTok{)}
\end{Highlighting}
\end{Shaded}

\begin{center}\includegraphics[width=0.5\linewidth]{vegan-slides_files/figure-beamer/NMDS23-1} \end{center}

\normalsize

\end{frame}

\begin{frame}[fragile]{Vectors in ordination space}

\scriptsize

\begin{Shaded}
\begin{Highlighting}[]
\NormalTok{dune.bray.ord.A1.fit <-}\StringTok{ }\KeywordTok{envfit}\NormalTok{(dune.bray.ord,dune.env$A1, }\DataTypeTok{permutations =} \DecValTok{1000}\NormalTok{)}
\NormalTok{dune.bray.ord.A1.fit}
\end{Highlighting}
\end{Shaded}

\begin{verbatim}

***VECTORS

       NMDS1   NMDS2     r2  Pr(>r)  
[1,] 0.99008 0.14052 0.3798 0.01698 *
---
Signif. codes:  0 '***' 0.001 '**' 0.01 '*' 0.05 '.' 0.1 ' ' 1
Permutation: free
Number of permutations: 1000
\end{verbatim}

\normalsize

\end{frame}

\begin{frame}[fragile]{Vectors in ordination space}

\scriptsize

\begin{Shaded}
\begin{Highlighting}[]
\KeywordTok{plot}\NormalTok{(dune.bray.ord, }\DataTypeTok{display =} \StringTok{"sites"}\NormalTok{, }\DataTypeTok{type =} \StringTok{"n"}\NormalTok{)}
\KeywordTok{points}\NormalTok{(dune.bray.ord,}\DataTypeTok{display =} \StringTok{"sites"}\NormalTok{, }\DataTypeTok{cex =} \DecValTok{2}\NormalTok{)}
\KeywordTok{plot}\NormalTok{(dune.bray.ord.A1.fit, }\DataTypeTok{add =} \OtherTok{TRUE}\NormalTok{)}
\KeywordTok{ordisurf}\NormalTok{(dune.bray.ord,dune.env$A1, }\DataTypeTok{add =} \OtherTok{TRUE}\NormalTok{)}
\end{Highlighting}
\end{Shaded}

\begin{center}\includegraphics[width=0.5\linewidth]{vegan-slides_files/figure-beamer/NMDS25-1} \end{center}

\normalsize

\end{frame}

\end{document}